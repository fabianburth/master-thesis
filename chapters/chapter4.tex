%%%%%%%%%%%%%%%%%%%%%%%%%%%%
% Master's Thesis          %	    										
% Fabian Burth, 2022-08-01 %
%%%%%%%%%%%%%%%%%%%%%%%%%%%%	

\npchapter{System Design}
This section describes the design of the \textit{Security and Compliance Data Lake}. It covers the conception of the data model, the selection of a database and the design of the the API. Thereby, it especially focuses on giving detailed information about the ideas and motives that lead to specific design decisions.

\section{Requirements}
Before actually going into the details of the systems design, the requirements have to be specified, since they are at the core of pretty much every design decision. 

\subsection{Functional Requirements}
\begin{table}[H]
	\begin{tabularx}{\linewidth}{|l|X|l|l|}
		\hline
		\rowcolor{lightgray}\multicolumn{4}{|l|}{\cellcolor{lightgray}{\textbf{A1: API}}} \\
		\hline
		\rowcolor{lightgray}
		\textbf{Ref.\#} & \textbf{Functionality} & \textbf{Category} & \textbf{Priority} \\
		\hline
	\end{tabularx}
	\caption[API Requirements]{API Requirements}
	\label{Tab:API Requirements}
\end{table}
The SCDL must be able to consume data from multiple different data sources over the internet.
The SCDL must be able to provide a SBOM of the components to the user.




\section{Data Model}

\section{Database}
\section{API}
Tool agnostic, how does the api look like to abstract away from different tools
