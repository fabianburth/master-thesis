%%%%%%%%%%%%%%%%%%%%%%%%%%%%
% Master's Thesis          %	    										
% Fabian Burth, 2022-08-01 %
%%%%%%%%%%%%%%%%%%%%%%%%%%%%

%\newpage
\thispagestyle{plain}
%\vspace*{\fill}
\chapter*{Abstract}
\addcontentsline{toc}{chapter}{Abstract}
%\fancyhead[L]{Abstract}
The importance of software is constantly growing and so is its complexity. Thus, large-scale enterprise software systems are usually not developed completely from scratch. Commonly required functionality like logging or serialization is often provided by already existing \textit{open source software (OSS)}. Hence, modern software systems are composed of several components. These components usually have individual version updates, vulnerabilities, and licenses. Version updates of a component may affect the compatibility with other components. Vulnerabilities in one specific component may compromise the security of the whole software system. Violating license agreements may lead to litigations due to copyright infringement. Furthermore, one must consider that each component itself can be composed of several other components.\par
So, maintaining and monitoring large-scale enterprise software systems is an important part of the \textit{Application Lifecycle Management (ALM)} and poses a considerable challenge to software companies. The common way to tackle these risks nowadays is by incorporating \textit{Software Component Analysis (SCA)} tools into the application development process. These tools analyze applications and retrieve information like vulnerabilities, licenses, and \textit{software bill of materials (SBOM)}. But this approach often still has its flaws. The information extracted by these tools is frequently treated like logs and hence of limited value for future usage. Additionally, in larger companies different development teams often come up with point-to-point solutions of integrating the tools tightly coupled to their development process.\par
The main objective of this thesis is the design and prototypical implementation of a \textit{Security and Compliance Data Lake (SCDL)}, which provides a standardized way of integrating even multiple different SCA tools into the development process and to store the extracted information. By offering an \textit{Application Programming Interface (API)}, it then should enable consumers to query this information on different levels of aggregation to answer questions that might not even have been known at the time the SCA was performed. A recent and popular example for such a question is \enquote{Which components contain log4j?}.\par
This \textit{Security and Compliance Data Lake} will build upon the \textit{Open Component Model (OCM)}, an open standard to describe the SBOM with so called\textit{ Component Descriptors}. These \textit{Component Descriptors} also describe how to access sources and resources. They thereby provide an entry point for the execution of SCA tools.

\vfil
\pagestyle{fancy}