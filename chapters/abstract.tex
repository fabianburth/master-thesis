%%%%%%%%%%%%%%%%%%%%%%%%%%%%
% Master's Thesis          %	    										
% Fabian Burth, 2022-08-01 %
%%%%%%%%%%%%%%%%%%%%%%%%%%%%

%\newpage
%\vspace*{\fill}
\titlespacing*{\chapter}{0pt}{0pt}{40pt}
\pagestyle{plain}
\chapter*{Abstract}
\addcontentsline{toc}{chapter}{Abstract}
%\fancyhead[L]{Abstract}
The importance of software is constantly growing and so is its complexity. Thus, large-scale enterprise software systems are usually not developed from scratch. Commonly required functionality like logging or serialization is often imported through already existing \emph{open source software}. Hence, modern software systems are composed of several components. These components usually have individual version updates, vulnerabilities, and licenses. Version updates of a component may introduce new vulnerabilities and licenses. Vulnerabilities in one specific component may compromise the security of the whole software system. Violating license agreements may lead to litigations due to copyright infringement. Furthermore, it must be considered that each component itself may be composed of several other components.\par
So, developing and maintaining secure large-scale enterprise software systems poses a considerable challenge to software companies. The state of the art approach to mitigate these risks is by incorporating compliance scanning tools into the CI/CD pipeline. These tools analyze the components within a software system and retrieve information like composition, vulnerabilities, and licenses. But this approach has its flaws. The components within a software system are commonly built through multiple different CI/CD pipelines maintained by individual development teams. These individual development teams frequently come up with technology specific point-to-point solutions of integrating the scanning tools into their CI/CD pipeline. Therefore, the information about the software systems of a company is frequently distributed over multiple development teams and therein again over multiple technology specific solutions.\par
The main objective of this thesis is the design and prototypical implementation of a central application to store this distributed information and to provide a consolidated view on all the software systems of a company. Through an API, the application should enable consumers to query this information to answer questions that might not even have been known at the time the compliance scan was performed. A recent and popular example for such a question is \enquote{Which systems contain Log4j?}.\par

\vfill
\titlespacing*{\chapter}{0pt}{50pt}{40pt}