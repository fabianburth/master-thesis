%%%%%%%%%%%%%%%%%%%%%%%%%%%%
% Master's Thesis          %	    										
% Fabian Burth, 2022-08-01 %
%%%%%%%%%%%%%%%%%%%%%%%%%%%%

\titlespacing*{\chapter}{0pt}{0pt}{40pt}
\pagestyle{plain}
\chapter*{Abstract (German)}
\addcontentsline{toc}{chapter}{Abstract (German)}
Die Relevanz und Komplexität von Software nimmt kontinuierlich zu. Daher werden Unternehmens-Software-Systeme in der Regel nicht von Grund auf entwickelt. Standardfunktionalitäten wie Logging oder Serialisierung werden häufig durch bereits bestehende Open-Source-Software importiert. Moderne Software-Systeme setzen sich dementsprechend aus diversen kleineren Software-Komponenten zusammen. Diese Komponenten haben dabei individuelle Versionswechsel, Verwundbarkeiten und Lizenzen. Verwundbarkeiten in einer einzelnen Komponente können das gesamte System kompromittieren. Verstöße gegen Lizenzvereinbarungen können zu teuren Gerichtsverfahren führen. Versionswechsel von Komponenten können weitere neue Verwundbarkeiten und Lizenzen einführen. Außerdem muss berücksichtigt werden, dass jede Komponente wiederum selbst aus weiteren Komponenten bestehen kann.\par
Die Entwicklung und Wartung sicherer Unternehmens-Software-Systeme stellt daher eine wesentliche Herausforderung für Softwareunternehmen dar. Der State-of-the-Art Ansatz, diese Risiken zu mitigieren, besteht im Integrieren von Compliance-Scanning Tools in die CI/CD-Pipeline. Diese Tools analysieren die Software-Komponenten eines Software-Systems und liefern Informationen zur Zusammensetzung, zu Verwundbarkeiten und zu Lizenzen. Dabei gibt es aber auch Probleme. Die Komponenten, aus denen sich ein Software-System zusammensetzt, werden häufig durch mehrere verschiedene CI/CD Pipelines erstellt, die von unterschiedlichen Entwicklungsteams gepflegt werden. Diese Entwicklungsteams entwickeln dabei oft technologiespezifische Punkt-zu-Punkt Lösungen, um die Scanning Tools in ihre CI/CD Pipeline einzubauen. Daher sind die Information über die Software Systeme eines Unternehmens oft über verschiedene Entwicklungsteams und in den Entwicklungsteams wiederum über verschiedene Technologien verteilt.\par
Das Hauptziel dieser Arbeit ist das Design und die Implementierung einer Applikation zur zentralen Speicherung dieser verteilten Informationen, die eine konsolidierte Sicht auf die Software Systeme eines Unternehmens ermöglicht. Durch eine API soll die Applikation die Abfrage von Informationen ermöglichen. Dabei sollen Fragen beantwortet werden können, die zum Zeitpunkt des Scans potentiell noch nicht bekannt waren. Ein aktuelles und populäres Beispiel für eine solche Frage ist: "Welche Systeme enthalten Log4j?".