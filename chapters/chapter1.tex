%%%%%%%%%%%%%%%%%%%%%%%%%%%%
% Master's Thesis          %	    										
% Fabian Burth, 2022-08-01 %
%%%%%%%%%%%%%%%%%%%%%%%%%%%%

\npchapter{Introduction}
\pagenumbering{arabic}

\section{Motivation}
\section{Goals}
The goal of this thesis is the design of a software solution 
\section{Environment}
This thesis is written in cooperation with SAP. As of today, with a total revenue of €27.34 billion, SAP is the third largest software company in the world after Microsoft and Oracle \cite{LargestSoftwareCompanies}. While also having gained some attention with \textit{Business to Customer (B2C)} products like the \textit{Corona-Warn-App}, its core products are \textit{Business to Business (B2B)} enterprise software solutions. Originally, SAP grew around its \textit{Enterprise Resource Planing (ERP)} system. Today, the company is also adopting \textit{Internet of Things (IoT)} technologies and \textit{Artificial Intelligence (AI)} to provide advanced analytics for its customers and to maximize the value of its software products \cite{AboutSAP}. Besides, SAP is also investing heavily to push its products and customers to the cloud. Therefore, the company maintains partnerships with Amazon, Microsoft, Google and Alibaba as infrastructure providers.\par 
The department this thesis is written with is developing and maintaining the \textit{SAP Gardener}. SAP Gardener is \textit{SAP's own managed Kubernetes} service which enables SAP itself as well as SAP customers to ship their applications to all of these different infrastructures using a unified deployment underlay. Since SAP Gardener offers the possibility to configure practically every detail, neither SAP nor its customers need to rely on the managed Kubernetes service of each hyperscaler with individual perks and restrictions, but can leverage the functionality of a fully configurable kubernetes cluster without actually having to fully configure it \cite{GardenerValueProposition}. 
   
\section{Structure of the Thesis}


