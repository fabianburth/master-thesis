%%%%%%%%%%%%%%%%%%%%%%%%%%%%
% Master's Thesis          %	    										
% Fabian Burth, 2022-08-01 %
%%%%%%%%%%%%%%%%%%%%%%%%%%%%

\pagestyle{fancy-style}
\npchapter{Introduction}
\pagenumbering{arabic}

\begin{quote}
	\emph{"[T]he trust we place in our digital infrastructure should be proportional to how trustworthy and transparent that infrastructure is"}\\
	Executive Order on Improving the Nation's Cybersecurity \cite{ExecutiveOrderSBOM}
\end{quote}

\section{Motivation}
The introductory quote above initially sounds pretty intuitive and self evident. Probably everyone in the IT industry would be able to agree on this. Yet, the panic and public outcry unleashed in the software industry after the vulnerability in Log4j was discovered, a popular and widely used logging library, goes to show how far the statement strives from reality. The vulnerability is rated with the highest severity possible since it enables \textit{Remote Code Execution (RCE)}, which in other words allows an attacker to run any code on the machine using Log4j \cite{Log4jVuln}. After the discovery, IT specialists all over the world had to identify which of their applications and systems were using vulnerable versions of the library before they could even start dealing with the vulnerability. This increases the response time and subsequently the risk exposure significantly. But what is it, that makes this such a difficult task?\par
Modern software systems are composed of numerous software components, such as Log4j, and these components may themselves be composed of other software components and so on. This adds several layers of complexity obfuscating the dependencies of a software system. The manufacturing industry has been dealing with such issues in their supply chains for years. But these companies usually maintain close relationships and contracts with their suppliers, perhaps even exchanging \textit{bills of materials (BOM)}. Thus, the companies can easily keep track of each individual part in their products by accumulating the BOMs of their suppliers and they will even be informed about problems affecting this parts by their suppliers. This is different from the software industry. Especially when it comes to \textit{open source software (OSS)}, companies do not have a contract with the supplier. On the contrary, they frequently might not even know the maintainer of the software component. Subsequently, they also will not be informed about issues such as vulnerabilities with the components.\par
The software industry has come up with approaches to deal with this issue and implemented measures to proactively monitor applications and detect known vulnerabilities in its components. With growing complexity of software and changing development and deployment landscapes, these approaches are not sufficient and fail to answer questions such as which systems use vulnerable versions of Log4j. Why and how they fail in these situations will be further examined in the course of this thesis. 
But as "[s]oftware is eating the world"\cite{MarcAndreessen} and thus, as companies, devices of our everyday life, cars and even medical and military devices rely on software, as the impact of software on our privacy and also physical security increases, so does the responsibility of companies providing software and the gap between trust and transparency becomes less acceptable.\par
In an effort to address this issue, the US government has also published an Executive Order on Improving the Nation's Cybersecurity, which will require every company supplying software to the government to provide an \textit{Software Bill of Materials (SBOM)} \cite{ExecutiveOrderSBOM,NTIASBOM}. As previously discussed, this is only possible if every company in the supply chain provides an SBOM for their software components. Therefore, the downstream impact of this Executive Order will most likely affect the entire industry.

\section{Goals}
The goals of this thesis aim to provide a research based practical contribution to closing the gap between the trust and transparency placed into our digital infrastructure \cite{ExecutiveOrderSBOM}.\par 
To make this more concrete, the main goal of this thesis is to develop a \textit{proof of concept (POC)} implementation of a \textit{Security and Compliance Data Lake(SCDL)}. As the name already suggests, this is an application for storing information about software components. Such information might be about the occurrence of known vulnerabilities, as already indicated in the previous paragraphs, but also about licenses or dependencies. This kind of information about software components will be called metadata from here on.\par
Thus, the goal is to design and develop a central application for storing and querying software metadata, which is able to cope with the complexity and requirements of modern development and deployment landscapes. Therefore improving the transparency of the entire software supply chain and enabling companies to answer questions such as which applications, systems or even landscapes in a company contain a certain vulnerability.\par
Thereby, this thesis also contributes to the knowledge concerning metadata stores and respective API design by answering following research question:
\begin{quote}
	\emph{What are the challenges arising during the development of a database application for the central metadata management of enterprise software systems and how may these challenges be solved?}
\end{quote}

Furthermore, the application may support companies in fulfilling the requirements resulting from the recently announced Executive Order. 

\section{Scope}
As the goal section already stated, this thesis is about designing and implementing a central application for software metadata management and its integration into modern development and deployment landscapes.\par
It is not about developing new ways of vulnerability detection. It is rather about monitoring systems and keeping track of already known vulnerabilities. This is just as important since a large amount of security incidents are caused by attackers exploiting such known vulnerabilities \cite{ModelBasedSecurityTesting}. Therefore, neither does this thesis compete with current security testing techniques which are used to find vulnerabilities in applications \cite{SecurityTesting}, nor does it discuss these in detail. From the perspective of the central metadata store, each security testing technique applied may be viewed as a potential data source. But storing known vulnerabilities is still just one, although probably the most popular, use case of this central metadata store. A major design goal was to keep it extensible, so that all kinds of metadata may be stored.  

\section{Environment}
This thesis is written in cooperation with SAP. As of today, with a total revenue of €27.34 billion, SAP is the third largest software company in the world after Microsoft and Oracle \cite{LargestSoftwareCompanies}. While also having gained some attention with \textit{Business to Customer (B2C)} products like the \textit{Corona-Warn-App}, its core products are \textit{Business to Business (B2B)} enterprise software solutions. Originally, SAP grew around its \textit{Enterprise Resource Planing (ERP)} system. Today, the company is also adopting \textit{Internet of Things (IoT)} technologies and \textit{Artificial Intelligence (AI)} to provide advanced analytics for its customers and to maximize the value of its software products \cite{AboutSAP}. Besides, SAP is also investing heavily to push its products and customers to the cloud. Therefore, the company maintains partnerships with Amazon, Microsoft, Google and Alibaba as infrastructure providers.\par 
The department this thesis is written with is developing and maintaining the \textit{SAP Gardener}. SAP Gardener is \textit{SAP's own managed Kubernetes service} which enables SAP itself as well as SAP customers to ship their applications to all of these different infrastructures using a unified deployment underlay. Since SAP Gardener offers the possibility to configure practically every detail, neither SAP nor its customers need to rely on the managed Kubernetes service of each hyperscaler with their individual perks and restrictions, but can leverage the functionality of a fully configurable kubernetes cluster without actually having to fully configure it \cite{GardenerValueProposition}. 
   
\section{Structure of the Thesis}
In order to achieve the goals laid out before, the thesis builds upon following structure:\\\\
\textbf{Foundation:} In this chapter, basic terminology and existing concepts in software metadata management are established.\\\\
\textbf{State of the Art:} In the previous motivation and goal sections, it is already mentioned that existing measures for monitoring software component metadata such as vulnerabilities are insufficient in some cases. Therefore, to further motivate this thesis, this chapter will briefly introduce the state of the art approach and point out its limitations. The identification of this problems is crucial for the successful design of the Security and Compliance Data Lake.\\\\
\textbf{Context at SAP Gardener:} As mentioned, this thesis is written with the SAP Gardener team. It is greatly inspired by the problems they are facing and by the approaches they already implemented to overcome limitations of the state of the art. Thus, this chapter introduces their proposed standard, the Open Component Model, and its capabilities. Based on that, their development and deployment landscape is explained. Finally, an architecture for integrating a central metadata store such as the one designed and developed in this work into their current environment and thereby overcoming all these limitations is presented.\\\\
\textbf{System Design:} The system design chapter covers the conception of the data model, the selection of a database and the design of the API. Thereby, it especially focuses on giving detailed information about the ideas and motives that lead to specific design decisions. Besides, alternatives that have been considered and the respective reasons to not follow through with them will be an essential part of this section.\\\\
\textbf{Implementation:} Although the implementation is a major and time intensive part of the thesis, this chapter is kept short. The code base has grown quite big and discussing it in detail would be out of scope. Thus, this chapter focuses on explaining the high level architecture and solely examines some particularly interesting parts of the implementation. To make this tangible, it is accompanied by some showcasing of the actual functionality.\\\\
\textbf{Results:} Finally, the design decisions and the implementation as well as the knowledge gained through this work is evaluated and the research question is revisited.


