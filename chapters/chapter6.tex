%%%%%%%%%%%%%%%%%%%%%%%%%%%%
% Master's Thesis          %	    										
% Fabian Burth, 2022-08-01 %
%%%%%%%%%%%%%%%%%%%%%%%%%%%%

\npchapter{Result}
After the technical and architectural design decisions are discussed in detail, this chapter finally presents the results, the actual functionality of the \emph{Security and Compliance Data Lake} as a result of this thesis. This includes an evaluation of the design decisions, the implementation and thereby, the actual value of this research based practical contribution to increase the trust and transparency placed in our digital infrastructure. To conclude this work, the research question is revisited, summarizing the thesis and highlighting the greatest challenges throughout the application design and development. 

\section{Presentation} \label{sec:Presentation}
This presentation section briefly introduces the infrastructure, the data model configuration and the data set of the application instance used to demonstrate the functionality, before showing queries answering interesting questions about software composition through the GraphQL API. After all, the \emph{Security and Compliance Data Lake} is a backend application providing its functionality through a standard interface called GraphQL API and returning its results in JSON, a machine readable format, limiting the possibilities for impressive visual showcasing.   

\subsection{Application Infrastructure}
Originally, the application was developed and tested on a \emph{Windows Subsystem for Linux 2 (WSL2)}. Thus, the database as well as the application were running there.\par
After the first PoC of the application was finished, SAP Gardener was used to set up a kubernetes cluster to host presentable versions of the application. Therefore, a simple development pipeline was configured with \emph{GitHub Actions} % Was ist GitHub Actions? und DockerHub? Eine Erklärung in einem Satz wäre ganz cool 😊
to automatically containerize the application and push the respective images into a \emph{DockerHub} repository. Furthermore, \emph{Kubernetes deployment files} were created to deploy the application itself as well as other Kubernetes resources such as an \emph{ingress} to expose the API. To deploy the Neo4j database, the corresponding \emph{helm chart} provided by Neo4j was leveraged. Additionally, in order to be %Auch wenn ich die anderen -ing-Formen durchgehen lassen habe, weil ich mir unsicher war ... hier fühlt sich die -ing-Form schon sehr falsch an 😉 
able to perform all operations to deploy and undeploy all Kubernetes resources at once, small bash scripts were set up.\par
In this Kubernetes cluster, all resources are deployed on a single \emph{worker node} with 4 CPUs and 16Gi of RAM. %Oben haste die Abkürzung CPU eingeführt, aber was ist mit RAM? Das ist auch eine Abkürzung, aber an der Stelle glaube ich nicht relevant. :D Gi würde ich evtl. trotzdem im Abkürzungsverzeichnis aufnehmen, da ich jetzt nicht wüsste, was das ist und es nicht so gängig wie GB ist
Furthermore, the cluster has 50Gi of persistent memory.

\subsection{Application Configuration \& Data}
The \emph{data model} configured through the \emph{Types API} for this application instance which is used for presentation purposes corresponds to SAP Gardeners \emph{Open Component Model} and the usage of \emph{Black Duck Binary Analysis} as scanning tool and sole \emph{data source}.\par
Thus, the \emph{data} created through the \emph{Consumption API} is provided by a \emph{Component Descriptor Repository} and \emph{Black Duck Binary Analysis}. The presentation application instance contains 22.204 \emph{nodes} and 97.335 \emph{relationships} between those \emph{nodes}. Figure \ref{fig:Neo4jGraph} shows a visual representation of a part of the graph\footnote{The Neo4j web interface becomes unresponsive and aborts the request if the amount of nodes and relationships to be rendered becomes to high. Therefore, it is not possible to show a visual representation of the entire graph here.} rendered by Neo4j to showcase the database's functionality and to give a better feeling for the data.\par

\begin{figure}[H]
	\centering
	\includegraphics[scale=1.0]{neo4j_graph}
	\caption[Neo4j Graph]{Neo4j Graph Representation \source{Own Representation}}
	\label{fig:Neo4jGraph}
\end{figure}

The lines connecting different nodes and thereby representing the relationships are barely visible and often not explicitly rendered, but Neo4j visualizes nodes that are highly interrelated close together leading to these visible clusters. The main clusters in the middle revolve around \emph{Component Versions} and contain all kinds of nodes. The larger clusters on the side are \emph{Package Versions} with an exceptionally high number of \emph{Vulnerabilities}. For example, the highlighted cluster on the bottom left revolves around a version of \emph{HashiCorp's Vault}, a software package for managing secrets and protecting sensitive data \cite{vault}.\\

Although this is difficult to arrange clearly, to give an even better understanding of how the data model manifests itself in form of nodes and relationships, figure \ref{fig:Neo4jComponentTree} zooms in on a specific component.

\begin{figure}[H]
	\centering
	\includegraphics[scale=0.70]{neo4j_graph_zoomed}
	\caption[Neo4j Component Tree]{Neo4j Component Tree Representation \source{Own Representation}}
	\label{fig:Neo4jComponentTree}
\end{figure}

The figure shows a part of the tree formed by the relationships of the \emph{Component} at the top.\footnote{This visualization rendered by Neo4j also shows the names of relationships along the lines and the names of some nodes within the circle. These are not intended to be readable in this figure. They are not important for the explanations and there is no way to zoom in close enough to keep them readable while still showing the whole tree.} To be more precise, the nodes that are highlighted by a thicker black border are the ones that are unfolded. Thus, for these nodes all related nodes are visualized.\par
The \emph{Component} at the top has only one relationship to a \emph{Component Version}. Thus, there is only one version of this \emph{Component} stored in the database. As the highlighted \emph{Component Version} straight under the \emph{Component} is unfolded, the figure shows that it references one other \emph{Component Version}. Moreover, it also references a \emph{Resource Version} and a \emph{Source Version}. Since the \emph{Source Version} is not unfolded, its related nodes are not shown. If it was unfolded, it would most likely open up a subtree similar to the one of the highlighted, and therefore unfolded, \emph{Resource Version}. The same is true for the \emph{Resource}. If it was unfolded, it would possibly show several more \emph{Resource Versions}. Besides the \emph{Resource}, the unfolded \emph{Resource Version} is especially related to several \emph{Package Versions}. The unfolded \emph{Package Version} is related to its other two aggregate levels, \emph{Package}, shown directly above itself and below the \emph{Resource Version}, and \emph{BDBA}, %Ich weiß bis jetzt immer noch nicht, was BDBA ist 😄 bitte im Abkürzungsverzeichnis hinzufügen
or rather, \emph{Native Package Version}. The relationships to the \emph{Resource Versions} on the right show that this particular \emph{Package Version} occurs in several more \emph{Resource Versions}. The topmost \emph{Resource Version} is even referenced by the \emph{Component Version} which is referenced by the unfolded \emph{Component Version}. Thus, there are two different paths through which this particular \emph{Package Version} is contained in the \emph{Component}. Furthermore, the \emph{Package Version} is related to, or rather has, a number of \emph{Vulnerabilities} and a \emph{License} which are both \emph{Info Snippets}. The \emph{Vulnerabilities} on the left even occur within another \emph{Package Version}.\par 
So, although this has been theoretically discussed before, this conveniently visualizes the paths that have to be traversed to answer questions related to software composition.


\subsection{Demonstration}
The \emph{data model configuration} is not really interesting, as it is just sending several schemas to the respective endpoints. There is not much more to show than what was already presented in section \ref{sec:Types API & Services} "Types API \& Services". Moreover, these schemas are SAP Gardener OCM specific and are even provisional for the PoC. For the productive use of the application, these will have to be adjusted in close cooperation with the corresponding stake holders within the SAP Gardener team.\par
The \emph{data upload} is handled by an \emph{Adapter} written in \emph{Python}. This \emph{Adapter} is tailored for the PoC use case. Therefore, the upload may be triggered manually by specifying a specific \emph{Component Version}. Consequently, all the \emph{References} to other \emph{Component Versions}, to \emph{Resource Versions} and  \emph{Source Versions} are resolved recursively and uploaded to the \emph{Security and Compliance Data Lake} in the correct order, so that the application does not attempt to create a relationship to an entity that might not exist yet. Furthermore, the \emph{Adapter} retrieves the scan results for respective \emph{Resource Versions} from the BDBA API and uploads them as well. The actual scans are triggered independently. This way, the results may be uploaded without performing a scan each time which is especially useful during development.\par
Finally, the \emph{data analysis} may be performed through the GraphQL API. To demonstrate the functionalities and capabilities of the \emph{Security and Compliance Data Lake}, the following paragraphs show how it may be used to answer questions about software composition. The application is currently not connected to a front end. Therefore, provisionally, the only way to retrieve specific data is to actually type the respective request body. To make this more comfortable, the \emph{Security and Compliance Data Lake} also hosts \emph{GraphiQL}, a Web IDE for GraphQL, offering syntax highlighting, auto-completion and automatic documentation \cite{graphiql}. This Web IDE is also used to run the following queries.\\

Throughout the thesis, the two most common questions are \emph{"Which Vulnerabilities are contained in a specific Component Version?"} and the reciprocal \emph{"Which Component Version contains a specific Vulnerability?}.\par
So, to address the first question, a \emph{Component Version} has to be picked. For this presentation, \emph{github.com/gardener/dashboard} version \emph{1.64.0} is used. This \emph{Component} is somewhat tangible as it represents actual releases of the \emph{SAP Gardener Dashboard}, a web application to monitor OCM components, which incidentally will prospectively also leverage the \emph{Security and Compliance Data Lake} to extend its functionality.

\begin{figure}[H]
	\centering
	\includegraphics[scale=0.52]{graphiql_component_vulns_query}
	\caption[GraphQL Query Vulnerabilities in Component]{Vulnerabilities in Component \source{Own Representation}}
	\label{fig:VulnsInComp}
\end{figure}

The screenshot from the GraphiQL Web IDE shows the GraphQL query on the left and the corresponding response on the right. The query specifies to return the \emph{Identity} property \emph{cve} and the \emph{Attribute} properties \emph{publish date} and \emph{cvss3\_score} %hier bitte wieder mit dem grauen Hintergrund hervorheben anstatt kursiv
for each \emph{Vulnerability} in the \emph{Component Version}. The right side shows that the query returns a JSON document containing an array of \emph{Vulnerabilities}. Thereby, the scrollbar indicates that the array contains several more \emph{Vulnerabilities} than visible in the screenshot. As briefly mentioned in section \ref{sec:Meta Data Model} "Meta Data Model", the entity type schemas themselve may be versioned. Since schema names in GraphQL have to be unique, a suffix, in above figure \emph{v1alpha1}, accounts for this schema versioning.\par 
To demonstrate how to answer the reciprocal query, the upper most \emph{Vulnerability} with the \emph{cve} of \emph{CVE-2020-8927} in above figure is used. 

\begin{figure}[H]
	\centering
	\includegraphics[scale=0.52]{graphiql_vulns_comp_query}
	\caption[GraphQL Query Components with Vulnerability]{Components with Vulnerability \source{Own Representation}}
	\label{fig:CompsWithVuln}
\end{figure}

The result in figure \ref{fig:CompsWithVuln} is scrolled down a bit, as indicated by the scrollbar, to show that the array of \emph{Component Versions} naturally also contains the \emph{Component Version} where the \emph{Vulnerability} was found in like already shown in figure \ref{fig:VulnsInComp}. The popular question \emph{"Which Deployments contain the Log4j vulnerability?"} could be answered correspondingly.\par
To further stress the capabilities of the GraphQL API, the following query shows how to request all \emph{Package Versions} contained in the \emph{github.com/gardener/dashboard} \emph{Component Version} and return the respective \emph{Licenses} at the same time.

\begin{figure}[H]
	\centering
	\includegraphics[scale=0.52]{graphiql_comps_pkg_license}
	\caption[GraphQL Query Packages in Component]{Packages in Component \source{Own Representation}}
	\label{fig:CompWithPkgs}
\end{figure}

Figure \ref{fig:CompWithPkgs} shows that the result is a JSON document containing an array of \emph{Package Versions} and again, nested in each \emph{Package Version}, an array of \emph{Licenses}. Thus, the queries may even be nested.\\

There is a lot more functionality that could be showcased. But as the above examples are sufficient to convey a basic understanding of the functionality and each further example takes up a lot of space, the following section transitions to an evaluation. 

\section{Evaluation}
To provide an evaluation of the PoC implementation of the \emph{Security and Compliance Data Lake} developed during this scientific work, the currently available functionality is compared to the requirements specified in section \ref{sec:Requirements}. Thereby, this section provides a holistic view, summing up how each requirement is fulfilled by the application or what has to be done in order to satisfy a currently unfulfilled requirement.\par
\emph{Requirement R.1} specifies that the application shall be able to consume and store metadata from multiple different data sources. This was a crucial aspect during data model and application design as also stressed in the corresponding chapters. The data model introduced \emph{classes of entity types} to express this extensibility and the application architecture exposes the \emph{Types API} to dynamically create and configure the actual data model tailored to the respective use case. The application thereby provides the possibility to not only configure different data sources for package information such as BDBA and Mend. As it also provides functionality to configure the \emph{component model} and theoretically support the option to use multiple \emph{component models} in parallel, requirement R.1 is exceeded.\par
\emph{Requirement R.2} stores metadata from different data sources without aggregation. The data model splits the package type into three aggregation levels, \emph{Package}, \emph{Package Version} and \emph{Native Package Version}. On the \emph{Native Package Version} aggregation level, the metadata for a package may be stored as provided by the respective data source without the necessity to merge, or rather aggregate, the data. Thus, requirement R.2 is fulfilled.\par
The \emph{Package Version} aggregation level is intended to store the aggregated data. As described in the meta data model, the \emph{Native Package Version} schema therefore has to specify the attributes specific to the data source, the \emph{native attributes} and the attributes that shall be aggregated, the \emph{attributes}. Although the current implementation creates this aggregate level implicitly, the merge has to be done manually. The API currently does not expose an operation to do this. Therefore, while the application architecture and design are prepared to fulfill this functionality, it currently is not available and is yet to be added. So requirement R.3 is not completely fulfilled. As the SAP Gardener team for whom the application was primarily developed, only uses actively the single data source BDBA for this kind of information at the moment, the actual implementation of the requirement had a lower priority for the PoC.\par
\emph{Requirement R.4}, providing an aggregation level to group sources and resources, is implemented through the component entity types. Thereby, the application bridges the gap between artifact metadata and deployment information.\par
\emph{Requirement R.5}, enabling to query data on different levels of aggregation, is probably the most visible requirement and is based on several of the previously mentioned requirements. As discussed in section \ref{sec:Query API & Services} "Query API \& Services" and also shown in the previous presentations, this is enabled by the GraphQL API. Thus, questions such as "Which vulnerabilities are contained in a specific resource or in a specific deployment?" %Wie bereits erwähnt, bitte die doppelten Hochkommata durch eine schöne Latex-Umgebung wie z.B. \glqq{}\grqq{} oder \quote{} ersetzen
may be answered effortlessly. The underlying complex graph traversals necessary to query this information are thereby completely transparent to the user.\par
As already discussed in section \ref{sec:Limitations and Problems} "Limitations and Problems" of the application's architecture, \emph{Requirement R.6} is currently not fulfilled by the PoC. Either an extension of the \emph{Types API} and respective services or a static extension of the schemas will have to be added to enable assessments.\par
\emph{Requirements R.7} to provide aggregation and filter functions and \emph{Requirement R.8} to enable querying the metadata in common SBOM formats are not implemented either. But based on the application's architecture, these functionalities may easily be added.\\

Regarding \emph{non-functional requirements} no specific boundaries were defined. Still it is mentioned several times that performance, especially of read operations, is relevant. Neither the amount of data in the presentation application instance, nor the compute resources of the cluster correspond exactly to the conditions of a productive application instance.\par
Anyway, as a point of orientation, the queries performed in section \ref{sec:Presentation} usually return in approximately 1-5 seconds, also depending on the internet connection and the amount of data requested.\par 
Theoretically, especially the performance of \emph{top-down queries}, thus, queries answering questions like "Which \emph{Vulnerabilities} are contained in a \emph{Component Version}?, should stay relatively constant even with far more nodes in the graph over all. \emph{Top-down} as \emph{Component Versions} build up a tree that is self-contained and aside of some discovered vulnerabilities (and possibly other \emph{Info Snippets}) usually stays relatively constant over time. Thus, corresponding to the explanations in section \ref{sec:GraphDB Theoretical Foundation} "Theoretical Foundations" of the graph database, the look up of the initial node is the only part of the query which may decrease significantly in performance. An important side note at this point, this is only true for \emph{top down} queries starting on \emph{Component Version} level and below, as the number of \emph{Component Versions}, and thus, the number of entire trees under a \emph{Component} will naturally grow significantly over time.\par
Generally, a similar concept applies to most of the \emph{bottom up queries}. So, answering questions like which \emph{Component Versions} contain a specific \emph{Package Version} naturally spans several of the \emph{Component Version} trees. But as new \emph{Component Versions} will generally switch to newer \emph{Package Versions}, this growth is limited in most cases.\par
Theoretically, the query performance should scale quite good, as most of them are likely limited to self-contained subgraphs.\\

To summarize this, the current PoC of the \emph{Security and Compliance Data Lake} implements almost all of the \emph{priority 1} requirements R.1 - R.5. Thereby, the application is already able to perform the most relevant queries and analysis about software composition and answer important questions such as \emph{"Which deployments contain the Log4j vulnerability?"} within seconds. As shown by the system design section, the decisions which lead the application's design and the application's architecture are based on thorough research. Thus, the application definitely poses a scientific and practical contribution to increase the transparency of our digital infrastructure.\par
Furthermore, the reasons for not fulfilling the other requirements are not actual hard limitations of the application's architecture, but merely the lack of time. So, the currently missing functionality will be added in future. %Kannst schon "will" nehmen, kann er ja eh nicht mehr prüfen 😄

\section{Research Question and Retrospective}
As common in software development, or rather in science in general, the complexity of a problem only really unfolds, once one actually starts trying to solve it. Correspondingly, the complexity to design and develop a sustainable solution for such a central software metadata store became much bigger than originally anticipated. This section sums up the thesis, highlighting the greatest challenges and concluding with a retrospective onto implementation and decisions.\\

First of all, the \emph{requirement elicitation} was an important part of this work. Due to the versatile ecosystem concerning software supply chain security and software metadata, this task proved challenging. Therefore, \emph{chapters 2 and 3} not only serve as foundation for the reader, but serve as valuable preparation for the entire work. %Ab hier aufpassen! Nicht das Inhaltsverzeichnis wieder runterschreiben, das ist ein beliebter Fehler.😉 Hier geht es eher um die speziellen Herausforderungen in den Kapiteln 2 und 3 als um das, was da drin steht, denn das weiß man ja schon (sollte man ja zumindest auch gelesen haben, wenn man an dieser Stelle ist) und wenn du beschreibst, worum's darin geht, wirklich nur für eine Argumentationskette
\emph{Chapter 2} provides an overview and basic understanding of the subject area and its existing standards and technologies. Thereon, \emph{chapter 3} introduces the current state of the art approach, pointing out its limitations, such as the coupling of CI and CD, the distribution of the software metadata over the software development life cycle and the resulting informational gap between artifacts and deployments.\\

\emph{Chapter 4} is focused on SAP Gardeners Open Component Model. The Open Component Model allows for decoupling CI from CD and provides a convenient and practice proved starting point for the design of a generic data model connecting artifact metadata with deployment metadata. Furthermore, \emph{chapter 4} provides a reference architecture of how to integrate the Open Component Model and the \emph{Security and Compliance Data Lake} into a modern development and deployment landscape, thereby stressing their practical applicability. As the Open Component Model and its tooling is itself relatively new, the documentation still lacks depth, explanations and extensive examples. Therefore, collecting the respective information required a lot of interviews and discussions with the responsible developers which were particularly difficult due to the abstract nature and the ambiguous terms of the Open Component Model. Consequently, this supposedly simple task of \emph{familiarizing with and explaining of existing concepts and tools}, turned out to be a real challenge.\\

\emph{Chapter 5} concludes the efforts of the previous chapters, leveraging the preparations to specify requirements and to design an entire software system. Thereby, the greatest challenges were the \emph{inconsistency of identities} and the \emph{volatility, or rather unpredictability, of the data model}.\par
The problem around artifact identities, thus, the ambiguity of artifact references regarding the actual technical artifacts, has been discussed in detail in the context of component models. But even more generally, in order to be able to reliably identify data delivered from different data sources as data about the same entity, this entity needs a \emph{globally unique identifier}. Although there are efforts to establish standards for identifying different entities, such as \emph{purl} for packages or the \emph{SPDX License Identifier} for licenses, these are rarely used by scanning tools at all or at least not consistently.\par
Considering the unpredictability of the data model, extending the entity relationship model with classes of entity types added some complexity to the explanations but was not particularly difficult itself. Dealing with such a generic data model during system design was a challenge as providing an architecture which enables enforcing the data model while maintaining the extensibility and flexibility to add new entity types and even configure the properties of entity types dynamically at runtime. Thereby, the greatest challenges include coming up with the \emph{Meta Data Model}, implementing code on the data source logic layer that \emph{constructs the statements in the database's query language, cypher, based on the concrete data model configured by the user} and developing \emph{APIs that dynamically adjust to this concrete data model}. So, if the user creates a new \emph{Info Snippet Type}, for example \emph{Build Information}, the \emph{Consumption API} subsequently also considers this entity type during \emph{validation}. Simultaneously, the \emph{Query API} automatically adds an option to query entities of type \emph{Build Information} based on their respective properties.\par
Besides, the research to determine the best database technology for the application was time intensive. But, as the decision to use a graph database was frequently questioned by colleagues and peers, the research has been proven to be worthwhile. Most of the skepticism may be traced back to unfamiliarity and has been mostly eliminated by the successful PoC of the \emph{Security and Compliance Data Lake}.\\

Other time and research intensive tasks that are not represented within the thesis are the proper containerization of the application and the configuration of the SAP Gardener Kubernetes cluster to run the presentation instance of the application, including authentication and encryption. Furthermore, implementing the adapter for the data upload based on existing coding of the SAP Gardener team was cumbersome. As the underlying code has undergone multiple incompatible changes during this work, the adapter had to be adjusted multiple times.\par
As the application utilizes several of the standards and their respective Go libraries, especially OpenAPI and GraphQL, way beyond their common, well-documented use cases, it was difficult to find suitable examples. Therefore, the development involved a lot of time intensive experimenting.\par
This and several other factors which significantly increased the cost of implementation may be traced back to the design decisions concerning the data model.  But, while the flexibility and extensibility of the data model and the consecutive introduction of the \emph{Meta Data Model} lead to an uncomfortable amount of complexity, it still seems necessary, even in retrospective. As an effort to reduce the complexity, it may be worthwhile to explore the possibilities to replace the \emph{Open API} schemas and tooling entirely by means of \emph{GraphQL}. The briefly mentioned initial prototype which had a rigid data model already leveraged \emph{Open API} which is why the application evolved in a way where such options were not explored thoroughly.\\

To conclude this scientific work: Compared to the state of the art approach, leveraging some kind of component model, such as the \emph{Open Component Model}, in combination with a central application for storing and querying software metadata, such as the \emph{Security and Compliance Data Lake}, as proposed in this thesis, is shown to undoubtedly be a significant practical advancement which can help companies in fulfilling the requirements resulting from the recently announced Executive Order. %Muss das wirklich ein Satz sein? Ist doch ziemlich lang 😄

\section{Outlook}
Regarding the \emph{Security and Compliance Data Lake}, the outlook includes the completion of the application by implementing the missing features mentioned during the evaluation. Thus, the aggregation, or rather merging, of data from different data sources, the assessments or triaging functionality and aggregation and filter functions will be added. Specifically within SAP Gardener, Mend shall be integrated as an additional data source, especially since it also provides information about package dependencies. Generally, there are a lot of data sources that may be added but could not all be covered in this work: Vulnerability databases such as the NVD, the Open Source Vulnerability (OSV) database or the Global Security Database (GSD), %OSV und GSD in Abkürzungsverzeichnis mit aufnehmen, danke 😊
or even more scanning tools such as Sonatype or Snyk\\
 
But to loosen the focus from the application and widen the scope to the general problem domain, the research area and the demand for solutions such as the \emph{Security and Compliance Data Lake} is rapidly growing. Even only during this scientific work, there were two publications which have the potential to change the ecosystem around software supply chain security and the general transparency of our digital infrastructure.\par
The first is the "Securing Open Source Act of 2022" released in September 2022 by the U.S. government \cite{SecuringOpenSourceAct}. This primarily discusses the implementation of a framework "for assessing the risk of open source software components" \cite{SecuringOpenSourceAct}. As this framework shall publicly be published within 1 year, it may be interesting to integrate respective assessments into the \emph{Security and Compliance Data Lake}. Due to its extensibility, this may easily be done as an additional type of \emph{Info Snippet}.\par
The second is the announcement of "GUAC" released in October 2022 by Google \cite{GUAC}. GUAC stands for \emph{Graph for Understanding Artifact Composition}. The article describes it as an application which "aggregates software security metadata into a high fidelity graph database" \cite{GUAC}. Goals include answering questions such as "Which parts of my organization's inventory is affected by new vulnerability X?" or "Where am I exposed to risky dependencies?" \cite{GUAC}. Thus, the foundational idea and underlying technology choice is actually quite similar to the \emph{Security and Compliance Data Lake}. Overall, the project is still in its early stages, but it might be worth watching.\\

Generally, developments such as the \emph{Security and Compliance Data Lake} at SAP and the \emph{Graph for Understanding Artifact Composition} at Google indicate that the message supplied by the introductory quote, that the trust we place in our digital infrastructure should be matched by how trustworthy and transparent that infrastructure is, seems to have successfully reached the consciousness of the industry leaders.

%Thema Bibliography: In (1) hast du ein anderes Datumsformat als sonst, wobei (2) dann deutsch ist und (3) wieder englisch. Am besten überall dieses format DD/MM/YYYY
%Bei (7) fehlt das Aufrufsdatum z.B. komplett; Bei (74) oder (78) würde ich versuchen, eine Überschrift zu nehmen, die das genauer beschreibt, anstatt das nur zu kopieren; Bei (82) fehlt mir auch noch bisschen mehr Info. Bei (86) ist die Klammer leer. Summa summarum, ein paar Kleinigkeiten 😉


