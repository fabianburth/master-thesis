%%%%%%%%%%%%%%%%%%%%%%%%%%%%
% Master's Thesis          %	    										
% Fabian Burth, 2022-08-01 %
%%%%%%%%%%%%%%%%%%%%%%%%%%%%

%Papierformat
\documentclass[a4paper, 12pt]{report}

% Seitenränder
\usepackage[a4paper,left=3cm,right=3cm,top=3cm,bottom=3cm]{geometry}

% Schriftpakete
\usepackage{tabularx}       % Dynamische Tabellen
\usepackage{multirow}       % Mehrere Zeilen untereinander zusammenfassen
\usepackage[utf8]{inputenc}	% Benutze UTF8 Zeichen
\usepackage[T1]{fontenc}	% Bessere Wort-Trennung
\usepackage[scaled]{helvet} % Skalierte Helvetica-Schriftart (serifenfrei)
\usepackage[gen]{eurosym}   % Eurosymbol '€'
\usepackage{underscore}     % Unterstriche mit '_' anstatt '\_' schreiben
\usepackage{lmodern}        % Latin Modern -> schönere Schrift in PDF-Files
\usepackage{microtype}      % Mikrotypografische Einstellungen verbessern
\usepackage[english]{babel} % Deutsche Silbertrennung
\usepackage{textcmds}       % Englische Anführungszeichen
\usepackage{seqsplit}

% Eineinhalbfacher Zeilenabstand
\usepackage[onehalfspacing]{setspace}

\usepackage{tocloft}

% Literaturverzeichnis
\usepackage[sorting=none, backend=bibtex, style=numeric]{biblatex}
\usepackage{xurl}

% Quellcode einbinden
\usepackage{ifthen}	  % IF-THEN Befehle
\usepackage{listings} % Quellcode anzeigen
\usepackage{textcomp} % Straight Single quotes in listings

% Mathematik
\usepackage{amsmath}  % Mathematische Funktionen
\usepackage{amsfonts} % Mathematische Schrift
\usepackage{amssymb}  % Mathematische Symbole
\usepackage{tikz}	  % Grafische Zeichnungen mit Linien, Punkten, Kurven, Rechtecken, ...
\usepackage[outline]{contour} % Pfeile fett machen

% Abkürzungsverzeichnis
\usepackage{acronym}
\usepackage{titlesec}

% Weitere Pakete
\usepackage{float}        % Floating objects wie figures oder tables
\usepackage{graphicx}     % Bilder/Grafiken
\usepackage{svg}		  % SVG (Skalierbare Vektorgrafiken) als Bildformat
\usepackage{pdfpages}     % PDF-Seiten/PDF-Dokumente einbinden
%\usepackage{filecontents} % Externe Files generieren
\usepackage{url}	  	  % Anzeige von URLs
\usepackage{fancyhdr} 	  % Kopf- und Fußzeile manipulieren
\usepackage{caption}      % Bildunterschrift
\usepackage{csquotes}     % Quotes
\usepackage{xpatch}
\usepackage{xcolor}
\usepackage{realboxes}

% Verweis-Einstellungen
\usepackage[colorlinks=true, % Blöcke ausblenden, Verweise farbig markieren
linkcolor=black, % Abkürzungs-Links in schwarz
citecolor=black, % Zitat-Links      in schwarz
urlcolor=blue]   % URL-Links        in blau
{hyperref}	     % Einbinden von URL

%Dokumenteneinstellung
% Kopf- und Fußzeile
%\fancypagestyle{fancy-style}{% Own style
	%\fancyhf{}% Clear header and footer
	%\fancyhead[C]{\slshape\nouppercase{\rightmark}}
	%	\fancyhead[LE,RO]{\slshape\nouppercase{\rightmark}}
	%	\fancyhead[LO,RE]{\slshape\nouppercase{\leftmark}}
	%\fancyfoot[C]{\sffamily\thepage}% Custom footer
	%\renewcommand{\headrulewidth}{0.4pt}% Line at the header visible
	%   \renewcommand{\footrulewidth}{0.4pt}% Line at the footer visible
%}

% Redefine the plain page style
%\fancypagestyle{plain}{%
%	\fancyhf{}%
%	\fancyfoot[C]{\sffamily\thepage}%
%	\renewcommand{\headrulewidth}{0pt}% Line at the header invisible
%	\renewcommand{\footrulewidth}{0pt}% Line at the footer invisible
%}

\pagestyle{fancy}

\titleformat{\chapter}[display]{\normalfont\sffamily\huge\bfseries}{\chaptertitlename\ \thechapter}{20pt}{\Huge}
\titleformat{\section}{\normalfont\sffamily\Large\bfseries}{\thesection}{1em}{}
\titleformat{\subsection}{\normalfont\sffamily\large\bfseries}{\thesubsection}{1em}{}
\titleformat{\subsubsection}{\normalfont\sffamily\normalsize\bfseries}{\thesubsection}{1em}{}

% Redefine content fonts in chapter, section, figure, table
\renewcommand{\cftchapfont}{\sffamily\bfseries}
\renewcommand{\cftsecfont}{}
\renewcommand{\cftfigfont}{Figure }
\renewcommand{\cfttabfont}{Table }

% Obere Trennlinie
\renewcommand{\headrulewidth}{0.4pt}   % obere Trennlinie
\setlength{\headheight}{14.5pt}        % Höhe der Kopfzeile

% Untere Trennlinie
%\renewcommand{\footrulewidth}{0.4pt}  % untere Trennlinie

% Einrücktiefe für Absätze
\setlength\parindent{24pt}			   % Einrück-Tiefe der ersten Zeile für alle folgenden Absätze

% Quellenverzeichnis
\addbibresource{chapters/literature.bib}
\renewcommand{\mkbibnamelast}[1]{\sffamily\MakeUppercase{#1}}
\xpatchbibdriver{misc} % Komma statt Punkt nach Titel
{\usebibmacro{title}\newunit}
{\usebibmacro{title}\printunit{\addcomma\space}}{}{}

% Starte neue Seite für jede neue Chapter
\let\stdchapter\chapter
\newcommand{\npchapter}{\newpage\stdchapter}

% Starte neue Seite für jede neue Section
\let\stdsection\section
\newcommand{\npsection}{\newpage\stdsection}

% Zeilenumbruch bei URLs korrekt gestalten
%\makeatletter 
%\g@addto@macro\UrlBreaks{ 
	%	\do\a\do\b\do\c\do\d\do\e\do\f\do\g\do\h\do\i\do\j 
	%	\do\k\do\l\do\m\do\n\do\o\do\p\do\q\do\r\do\s\do\t 
	%	\do\u\do\v\do\w\do\x\do\y\do\z\do\&\do\1\do\2\do\3 
	%	\do\4\do\5\do\6\do\7\do\8\do\9\do\0\do\-\do\/\do\_\do\.} 
\expandafter\def\expandafter\UrlBreaks\expandafter{\UrlBreaks\do\a%
	\do\b\do\c\do\d\do\e\do\f\do\g\do\h\do\i\do\j\do\k\do\l\do\m\do\n%
	\do\o\do\p\do\q\do\r\do\s\do\t\do\u\do\v\do\w\do\x\do\y\do\z\do\&}

% Grafikpfad setzen
\graphicspath{{images/}}

% Sprache
\selectlanguage{english}

% Visio-Grafiken ohne Rand einbinden
\newcommand{\includevisio}[2][]{\includegraphics[clip, trim=0.7mm 0.7mm 0.7mm 0.7mm, #1]{#2}}

% Bildunterschriften - Quelle
\newcommand{\quelle}{\\ Source: }

\definecolor{codeGray}{rgb}{0.9,0.9,0.9}

% Setze als Programmiersprache HTML
\lstdefinestyle{myStyle}{
	breaklines=true,
	numbers=none, 
	escapeinside={\%*}{*)},
	basicstyle=\footnotesize\ttfamily,
	showstringspaces=false,
	keepspaces=true,
	identifierstyle=\color{black},
	backgroundcolor=\color{codeGray},
}
\lstset{style=myStyle, language = HTML}

\colorlet{punct}{red!60!black}
\definecolor{delim}{RGB}{20,105,176}
\colorlet{numb}{magenta!60!black}
\lstdefinelanguage{json}{
	upquote=true,
	literate=
	*{:}{{{\color{punct}{:}}}}{1}
	{,}{{{\color{punct}{,}}}}{1}
	{\{}{{{\color{delim}{\{}}}}{1}
	{\}}{{{\color{delim}{\}}}}}{1}
	{[}{{{\color{delim}{[}}}}{1}
	{]}{{{\color{delim}{]}}}}{1},
}
\lstdefinelanguage{java}{
	upquote=true,
	literate=
	*{;}{{{\color{DCLred}{;}}}}{1}
	{[}{{{\color{delim}{[}}}}{1}
	{]}{{{\color{delim}{]}}}}{1},
	morekeywords={encodedString},
	keywordstyle={\color{black}},
	morekeywords=[2]{String, byte, binary}, 
	keywordstyle=[2]{\color{delim}}
}

\colorlet{DCLmagenta}{punct}
\colorlet{DCLred}{red}
\colorlet{DCLblue}{blue}
\lstdefinelanguage{dcl}{
	upquote=true,
	literate =
	*{;}{{{\color{DCLred}{;}}}}{1}
	{0}{{{\color{DCLblue}{0}}}}{1}
	{1}{{{\color{DCLblue}{1}}}}{1}
	{2}{{{\color{DCLblue}{2}}}}{1}
	{3}{{{\color{DCLblue}{3}}}}{1}
	{4}{{{\color{DCLblue}{4}}}}{1}
	{5}{{{\color{DCLblue}{5}}}}{1}
	{6}{{{\color{DCLblue}{6}}}}{1}
	{7}{{{\color{DCLblue}{7}}}}{1}
	{8}{{{\color{DCLblue}{8}}}}{1}
	{9}{{{\color{DCLblue}{9}}}}{1}
	{[}{{{\color{DCLred}{[}}}}{1}
	{]}{{{\color{DCLred}{]}}}}{1},
	morekeywords={POLICY, DEFAULT, FUNCTION, TEST, SCHEMA},
	keywordstyle={\color{delim}},
	morekeywords=[2]{GRANT, IN, ON, WHERE, AND, OR, LIKE, BETWEEN, USE, IS, NOT, NULL, RETURN, DENY, INPUT, Number, String}, 
	keywordstyle=[2]{\color{DCLred}}
}


\makeatletter
\xpretocmd\lstinline{\Colorbox{codeGray}\bgroup\appto\lst@DeInit{\egroup}}{}{}
\makeatother

% Fortführende Absätze einrücken
\newcommand{\absatz}{\noindent\hspace*{12pt}}

\newcommand{\hash}[1]{\ttfamily\seqsplit{#1}}
\newcommand{\colorred}[1]{\textcolor{red}{#1}}
\newcommand{\colorblue}[1]{\textcolor{blue}{#1}}
\newcommand{\colorgreen}[1]{\textcolor{teal}{#1}}

\newcommand\tab[1][1cm]{\hspace*{#1}} % \tab-command